\subsection{Frequency tables}
A frequency distribution is a table that shows ``classes`` or ``intervals`` of data 
entries with a count of the number of entries in each class. The frequency $f$ of 
a class is the number of data entries in the class. Each class will have a 
``lower limit``' and an ``upper limit`` which can be interpreted as the 
lowest and highest numbers in each class. The “class width” is defined as the 
distance between the lower limits of consecutive classes. Before constructing a 
frequency table, some consideration should be given about the range of values in 
the dataset. In situations where there are to many class intervals, the likelihood 
of reducing the bulkiness of the data is highly unlikely. On the other hand, if 
the total number of classes is minimal, then the shape of the distribution itself 
cannot be successfully determined. Generally, for most datasets $6–14$ intervals 
is considered an ideal benchmark. However, this should not be interpreted as the 
defacto standard as a lot depends on the dataset itself. With that being said, 
the following are a few general guidelines one can follow when constructing a 
frequency table. 

\begin{enumerate}
    \item The ideal number of classes can be determined or approximated by the 
    formulas: 
    \begin{align}
        C = 1 + 3.3\log{n} %\\ \eqname{Sturge's Rule}
    \end{align}
    \begin{align}
        C = \sqrt{n} %\\ \eqname{Square Root Choice Rule}
    \end{align}
    where $n$ is the total number of observations in the dataset. 
    \item Calculate the range of the data by finding the minimum and maximum data 
    values. 
    \item  Using the range, find the width of the classes which can be determined
    using the formula:
    \begin{equation}
        \mbox{Class Width} = \frac{\mbox{range}}{\mbox{number of classes}}
    \end{equation}
    \item To find the class limits use the minimum data entry as the lower limit 
    of the first class. Then to get the lower limit of the next class, add the 
    class width. Continue until you reach the last class. Then find the upper 
    limits of each class.
\end{enumerate}

\includegraphics[height=4cm]{example-image-a} 

\subsection{Frequency Distribution Graphs}
A frequency distribution graph is a diagrammatic illustration of the information 
in the frequency table. 

\subsubsection{Histogram}
A histogram is a graphical representation of the variable of interest in the 
$X$ axis versus the number of observations (frequency) in the $Y$ axis. Percentages 
can be used if the goal is to compare two histograms with a different number of 
subjects. Typically, a histogram is used to depict the frequency when data is 
measured against an interval or ratio scale. Coincidently, there is a striking
resemblance between a bar diagram and a histogram. However, they are nothing alike 
with three important distinctions between them. First off, in a histogram, there 
is no gap between the bars as the variable is continuous. A bar diagram will 
oftentimes have a noticeable amount of space between the bars.
Secondly, in histograms the width of the bars have meaning and do not need to be 
of equal length as this depends on the class interval. Whereas in a bar diagram all
the bars widths are equal in length. Finally, the area of each bar corresponds 
to the frequency in a histogram whereas in a bar diagram, it is the height. Figure
XX ... 

\includegraphics[height=4cm]{example-image-a} 

\subsubsection{Frequency Polygon}
A frequency polygon is very similar to a histogram. In fact, they are almost 
identical except that frequency polygons can be used to compare sets of data or 
to display a cumulative frequency distribution. A cumulative distribution is a 
form of a frequency distribution that represents the sum of a class and all the
classes below it. They are extremely useful when you need to determine the frequency 
up to a specific threshold or to easily compare two frequency distributions quickly.
Visually, there is also a slight difference where histograms tend to have rectangles 
while a frequency polygon resembles a line graph. Constructing a frequency polygon 
is done by connecting all midpoints of the top of the bars in a histogram by a 
straight line without displaying the bars. Also, when the total frequency is large 
and the class intervals are narrow, the frequency polygon becomes a smooth curve 
known as the frequency curve. 

\includegraphics[height=4cm]{example-image-a} 

\subsubsection{Box and whisker plot}
This graph, first described by Tukey in 1977, can also be
used to illustrate the distribution of data. There is a vertical
or horizontal rectangle (box), the ends of which correspond
to the upper and lower quartiles (75th and 25th percentile,
respectively). Hence the middle 50\% of observations are
represented by the box. The length of the box indicates the
variability of the data. The line inside the box denotes the
median (sometimes marked as a plus sign). The position of
the median indicates whether the data are skewed or not.
If the median is closer to the upper quartile, then they are
negatively skewed and if it is near the lower quartile, then
positively skewed.
The lines outside the box on either side are known as whiskers
[Figure 3]. These whiskers are 1.5 times the length of the
box, i.e., the interquartile range (IQR). The end of whiskers is
called the inner fence and any value outside it is an outlier. If
the distribution is symmetrical, then the whiskers are of equal
length. If the data are sparse on one side, the corresponding side
whisker will be short. The outer fence (usually not marked)
is at a distance of three times the IQR on either side of the
box. The reason behind having the inner and outer fence at
1.5 and 3 times the IQR, respectively, is the fact that 95\% of
observations fall within 1.5 times the IQR, and it is 99\% for
3 times the IQR.[5]

\includegraphics[height=4cm]{example-image-a} 

\subsubsection{Bubble Chart}
A Bubble Chart is a multi-variable graph that is a cross between a Scatterplot 
and a Proportional Area Chart. Like a Scatterplot, Bubble Charts use a Cartesian 
coordinate system to plot points along a grid where the X and Y axis are separate 
variables. However. unlike a Scatterplot, each point is assigned a label or 
category (either displayed alongside or on a legend). Each plotted point then 
represents a third variable by the area of its circle. Colours can also be used 
to distinguish between categories or used to represent an additional data variable. 
Time can be shown either by having it as a variable on one of the axis or by 
animating the data variables changing over time.

Bubble Charts are typically used to compare and show the relationships between 
categorised circles, by the use of positioning and proportions. The overall 
picture of Bubble Charts can be used to analyse for patterns/correlations. Too 
many bubbles can make the chart hard to read, so Bubble Charts have a limited 
data size capacity. This can be somewhat remedied by interactivity: clicking or 
hovering over bubbles to display hidden information, having an option to 
reorganise or filter out grouped categories. Like with Proportional Area Charts, 
the sizes of the circles need to be drawn based on the circle’s area, not its 
radius or diameter. Not only will the size of the circles change exponentially, 
but this will lead to misinterpretations by the human visual system.

\includegraphics[height=4cm]{example-image-a} 

\subsubsection{Multi-set Bar Chart}
Also known as a Grouped Bar Chart or Clustered Bar Chart. This variation of a Bar 
Chart is used when two or more data series are plotted 
side-by-side and grouped together under categories, all on the same axis. Like a 
Bar Chart, the length of each bar is used to show discrete, numerical comparisons 
amongst categories. Each data series is assigned an individual colour or a 
varying shade of the same colour, in order to distinguish them. Each group of 
bars are then spaced apart from each other. The use of Multi-set Bar Charts is 
usually to compare grouped variables or categories to other groups with those 
same variables or category types. Multi-set Bar Charts can also be used to compare 
mini Histograms to each other, so each bar in the group would represent the 
significant intervals of a variable. The downside of Multi-set Bar Charts is that 
they become harder to read the more bars you have in one group.

\includegraphics[height=4cm]{example-image-a} 