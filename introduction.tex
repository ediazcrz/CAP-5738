In today's digital landscape, data has become complex and bulky as it continues to 
grow from independent sources. In fact, there is so much data available that the 
term Big Data has become mainstream across industries with data analytics as a 
driving force behind innovation. For companies hoping to leverage datasets, fully
understanding them is key to effectively create strategic advantages in their respective 
industries. To accomplish this after the initial data collection phase organizing 
the data into a meaningful form so that a trend, if any, emerging out of the data 
is a critical step. 

One of the most common methods used for organizing data are frequency distributions.
A frequency distribution which is an overview of all the distinct values in some 
variable and the number of times they occur is a standard visualization technique. 
In practice frequency distributions are most commonly used to summarize categorical 
variables in datasets. If constructed well a frequency distribution is sometimes 
enough to make a detailed analysis of the structure of a population with respect 
to a given characteristic. Furthermore, one can easily spot whether observations 
are high or low and concentrated in one area or spread out across the entire scale. 

