The main advantages of certain visualizations for frequency distributions is highly 
dependent on the dataset and use case. Frequency tables are simple and easy to 
use and should be constructed first over other more complex visualizations. 
However, given that their usefulness is highly dependent on data complexity and 
size caution should be taken to avoid potentially concealing valuable information. 
Graphs such as histograms are a nice upgrade from frequency tables but should not 
be the only visualization relied upon. In fact, a combination of multiple graphs 
like the ones mentioned is ideal to fully illustrate a datasets features and flaws. 
Ultimately, in practice frequency distribution graphs have the most to offer during 
the exploratory phases of a dataset and should be used as much as possible before 
complex algorithms are designed to further process data. 