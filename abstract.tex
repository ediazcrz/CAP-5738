% Author: Ernesto Diaz
% Panther ID: 4890534
A frequency distribution is a list, table or graph that organizes all distinct
values of some variable within a given interval. They are mostly used to summarize
categorical variables and organize data into a meaningful form so that a trend, 
if any can easily be spotted. In practice, frequency distributions grant researchers 
and stakeholders alike the opportunity to glance at an entire dataset conveniently. 
They can highlight whether observations are high, low, concentrated in one area or 
spread out across an entire scale. Understanding how to display frequency distributions
and some of the options available is a critical first step in fully comprehending 
a dataset.  