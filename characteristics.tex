There are four important characteristics of frequency distributions.

\subsection{Measures of Central Location}

Oftentimes when frequency distribution data is graphed it is common for a significant
amount of data points to cluster around a central value. This clustering is known 
as the central location or central tendency of a frequency distribution. Once the 
value that a distribution centers around is known, it can be used to further 
characterize the rest of the data in the distribution. To calculate a central value 
several methods exist with each method producing somewhat of a different value. 
Collectively these methods can be refereed to as Measures of central location and 
the three most commonly used are:

\begin{enumerate}
    \item Mean: the sum of all values divided by the total number of values.
    \item Median: the middle number in an ordered data set.
    \item Mode: the most frequent value.
\end{enumerate}

These three measures are best used in combination with one another. This is because 
they have complementary strengths and limitations. The mode can be used for any 
level of measurement, but it is most meaningful for nominal and ordinal values.
The median can only be used on data that exhibits some type of order and the mean 
can only be used on interval and ratio values of measurement because it requires 
equal spacing between adjacent values or scores in the scale. Most of the time 
depending on the dataset, only one or two of these measures are applicable at any 
given time.

\subsection{Measures of Dispersion}

A second property of frequency distributions is dispersion or variation, which is 
the spread of a distribution out from its central value. The dispersion of a frequency 
distribution is independent of its central location. Figure ?? illustrates this 
fact, by showing the graph of three theoretical frequency distributions that have 
the same central location but different amounts of dispersion. Some of the more common
measures of dispersion that are used include the following:

\begin{enumerate}
    \item Range: the difference between the largest and the smallest observation 
    in the dataset.
    \item Interquartile Range: the difference between the 25\textsuperscript{th} and 
    75\textsuperscript{th} percentile (also called the first and third quartile).
    \item Standard Deviation: Measures the spread of data about the mean. 
\end{enumerate}

Much like measures of central location, measures of dispersion have their own
strengths and weaknesses. The biggest advantage of the range is that it is easy 
to calculate but has many disadvantages to be aware of. For instance, it is very 
sensitive to outliers and does not use all the observations in a data set. Additionally, 
it is more informative to provide the actual minimum and the maximum values rather 
than providing the range a singular value.

The interquartile range has an important advantage given that it can be used as 
a measure of variability if there are extreme values in the dataset that are not 
recorded exactly. This leads to the other advantageous feature that the interquartile 
range is not affected by extreme values. However, the main disadvantage of the 
interquartile range is that it is not amenable to mathematical manipulation.

Standard Deviation (SD) is perhaps the most famous and widely used measure for dispersion
calculation. The reason why is because if the observations are from a normal distribution, 
then, 68\% of observations lie between mean $\pm1$ SD, 95\% of observations lie 
between mean $\pm{2}$ SD and 99.7\% of observations lie between mean $\pm{3}$ SD. 
The other advantage of SD is that along with the mean it can be used to detect 
skewness. However, its biggest disadvantage is its inability to be used as an 
appropriate measure of dispersion for skewed data.

\subsection{Skewness}

Skewness is a measure of symmetry, or more precisely, the lack of symmetry. A 
distribution is symmetric if it looks the same to the left and right of the center 
point. The skewness for a normal distribution is zero, and any symmetric data 
should typically have a skewness near zero. Negative values for the skewness 
indicate that a dataset has a majority of its data points skewed left while positive 
values indicate a majority of data points are skewed right. For context when we
say "skewed left", we mean that the left tail is long relative to the right tail. 
Similarly, "skewed right" means that the right tail is long relative to the 
left tail. We can define the skewness with the formula:
\begin{equation}
    Skewness = \sum{}{} \frac{(X_i-\bar{X})^3}{ns^3}
\end{equation}
where $n$ is the sample size, $X_i$ is the $i\textsuperscript{th}$ $X$ value, $X$ 
is the average and $s$ is the sample standard deviation. However, most software 
tools such as Microsoft Excel take into account the sample size as well. Therefore,
we can slightly modify the formula to the following:
\begin{equation}
    \begin{split}
        Skewness & = \frac{n}{(n-1)(n-2)}\sum{}{} \frac{(X_i-\bar{X})^3}{s^3} \\
        & = \frac{n}{s^3(n-1)(n-2)}(S_{above} - S_{below})
    \end{split}
\end{equation}
In practice, as the sample size increases the difference in the results that these two 
formulas produce is relatively small so either one can be used with confidence. 

%\subsection{Kurtosis}
%\begin{enumerate}
%    \item Measures of central tendency and location (mean, median, mode).
%    \item Measures of dispersion (range, variance, standard deviation).
%    \item The extent of symmetry/asymmetry (skewness).
%    \item The flatness or peakedness (kurtosis).
%\end{enumerate}
