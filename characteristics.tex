There are four important characteristics of frequency distributions.

\subsection{Measures of Central Location}

Oftentimes when frequency distribution data is graphed it is common for a significant
amount of data points to cluster around a central value. This clustering is known 
as the central location or central tendency of a frequency distribution. Once the 
value that a distribution centers around is known, it can be used to further 
characterize all of the data in the distribution. To calculate a central value 
several methods exist with each method producing somewhat of a different value. 
Collectively these methods can be refereed to as Measures of central location and 
the three most commonly used are:

\begin{enumerate}
    \item Mean: the sum of all values divided by the total number of values.
    \item Median: the middle number in an ordered data set.
    \item Mode: the most frequent value.
\end{enumerate}

These three measures are best used in combination with one another. This is because 
they have complementary strengths and limitations. The mode can be used for any 
level of measurement, but it is most meaningful for nominal and ordinal values.
The median can only be used on data that exhibits some type of order and the mean 
can only be used on interval and ratio values of measurement because it requires 
equal spacing between adjacent values or scores in the scale. Most of the time 
depending on the dataset, only one or two of these measures are applicable in real 
world uses cases.

\subsection{Measures of Dispersion}

A second property of frequency distributions is variation or dispersion, which is 
the spread of a distribution out from its central value. Figure 3.3 shows the 
example where three curves/distributions have the same dispersion. Some of the 
measures of dispersion that we often use are the range, variance, and the standard 
deviation. The dispersion of a frequency distribution is independent of its central 
location. This fact is illustrated by Figure 3.4 which shows the graph of three 
theoretical frequency distributions that have the same central location but 
different amounts of dispersion
\subsection{Skewness}
\subsection{Kurtosis}
%\begin{enumerate}
%    \item Measures of central tendency and location (mean, median, mode).
%    \item Measures of dispersion (range, variance, standard deviation).
%    \item The extent of symmetry/asymmetry (skewness).
%    \item The flatness or peakedness (kurtosis).
%\end{enumerate}
